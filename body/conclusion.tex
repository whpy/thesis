\chapter{总结与展望}

\section{全文总结}
本文主要研究内容可分为两部分,第一部分研究了侧壁作用对槽道湍流带的影响,第二部分对神经网络应用于湍流带瞬态性预测的方法做了初步的验证。主要结论如下:

在第一部分中,本文采用直接数值模拟技术,借助Nektar++通过高阶谱元法对NS方程进行离散。通过植入特殊局部体积力在生成单个湍流带,并在4个不同雷诺数下研究湍流带的碰壁过程。本文确定了临界雷诺数$Re_{cr} \simeq 975~1000$,雷诺数在此临界值之下时,湍流带碰壁后头部消失且无法进一步自持;反之,湍流带头部虽仍然消失但却能在槽道中长期存活。通过进一步分析雷诺数大于以及小于临界值(即Re=1050,Re=750)两种情况下湍流与侧壁互相作用的详细过程,本文发现头部速度条纹的角度变化和湍流度的变化受侧壁影响最大。通过在泊肃叶流种分别放缩湍流带头部展向速度和流向速度的大小,本文发现角度变化对于展向速度型的变化更敏感。因此可合理地得出推论:侧壁效应通过影响展向速度型来影响头部流动不稳定性最终使头部消失。

在第二部分中,本文阐述了关于如何训练ESN神经网络预测同样具有瞬态性的MFE流动的衰减的相关工作。分析了MFE流动的瞬态性和湍流带瞬态性的相似性,本文尝试将该工作迁移道湍流带的衰减预测上。首先本文通过POD方法对流场进行降维并得到主要模态和对应的模数,随后分别应用了两种神经网络,分别是与文中一致的改良ESN网络以及应用广泛的LSTM网络,对模数的时间序列进行学习并预测预测。得出的结论是这两种神经网络都不能很顺利的预测出符合物理过程的衰减。ESN预测出了完全丢失关键力学特征的衰减过程,这证明它的结果是不可靠的;而LSTM无法预测出衰减,其预测出了一个近似周期的解且符合关键的力学特征,这说明LSTM学习到了关键的力学演化因素然而没能学习出深层的瞬态特性。

\section{本文的创新点}
1、本文采用了灵活度更高和精度更高的谱元法研究湍流带的力学行为。与过去相似的研究相比,本文中采取了更大的带侧壁槽道,确保了生成的湍流带在演化足够长的时间后才与侧壁碰撞。这使本文确定的临界雷诺数取值范围以及关于侧壁效应作用机理的推论更具可靠性。首次给出了临界雷诺数的大致范围并探讨了低雷诺数下碰撞时湍流带的衰减机制。

2、探讨了运用神经网络预测湍流带瞬态特性的可行性,发现了相关方法推广到湍流带寿命预测任务上的有限性。

\section{研究展望}
在未来的研究中我们考虑可在以下方面进一步研究:

关于槽道湍流带的侧壁影响:

1、本文中所采用的侧壁模型是最经典的无滑移周期槽道流动模型,边界条件相对简单。而在将来的相关研究中,可以进一步的考察复杂边界条件下湍流带与边界相互作用的过程与结果,如滑动边壁,带粗糙度的边壁,考虑进出口效应的边界条件等。

2、本文仅研究了湍流带在侧壁作用下的影响。而在后续的研究中可以进一步考察电磁效应对湍流带的影响,或者电磁效应以及侧壁效应对湍流带的联合影响。

关于湍流带的瞬态性预测的研究,本文初步探索了神经网络是否能通过湍流带在自持段的数据,捕捉学习到湍流带的瞬态特性。本文的主要工作大部分参考了Anton在文中提到的工作流程,然而并未能取得文中提到的同样出色的效果。因此我们考虑在将来的工作中从以下方面进行改进:

1、改进模态降维方式。本文中使用的是POD降维。然而POD降维原理十分简单且物理意义明确,从可解释性的角度来看这是很理想的特性。然而有一种可能是过于明确的物理意义造成了数据处理的偏见,控制瞬态性的关键要素或许与流向能量无关,而根据能量占比选取的主模态中有可能没有包含该关键要素。即POD作为压缩方法存在泛化性不足的可能性。因此在后续的研究中,我们可以尝试更新的基于机器学习的压缩算法比如VAE等。

2、尝试更换神经网络。LSTM发展时间更久,落地的应用更加成熟。然而近年来学界提出了许多更优秀的时序预测网络,比如GPT系列基于的transformer架构,以及基于PINN的时间序列预测神经网络等。