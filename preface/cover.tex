% % !Mode:: "TeX:UTF-8"

% %%  可通过增加或减少 setup/format.tex中的
% %%  第274行 \setlength{\@title@width}{8cm}中 8cm 这个参数来 控制封面中下划线的长度。

% \cheading{天津大学~2016~届本科生毕业论文}      % 设置正文的页眉,需要填上对应的毕业年份
% \ctitle{基于顾客有限理性预期的定价与供应链结构}    % 封面用论文标题,自己可手动断行
% \caffil{管理与经济学部} % 学院名称
% \csubject{工业工程}   % 专业名称
% \cgrade{2012~级}            % 年级
% \cauthor{秦昱博}            % 学生姓名
% \cnumber{3012209017}        % 学生学号
% \csupervisor{杨道箭}        % 导师姓名
% \crank{副教授}              % 导师职称

% \cdate{\the\year~年~\the\month~月~\the\day~日}

% \cabstract{
% 中文摘要一般在~400~字以内,简要介绍毕业论文的研究目的、方法、结果和结论,语言力求精炼。中英文摘要均要有关键词,一般为~3~—~7~个。字体为小四号宋体,各关键词之间要有分号。英文摘要应与中文摘要相对应,字体为小四号~Times New Roman,详见模板。
% }

% \ckeywords{关键词~1;关键词~2;关键词~3;……;关键词~7(关键词总共~3~—~7~个,最后一个关键词后面没有标点符号)}

% \eabstract{
% The upper bound of the number of Chinese characters is 400. The abstract aims at introducing the research purpose, research methods, research results, and research conclusion of graduation thesis, with refining words. Generally speaking, both the Chinese and English abstracts require the keywords, the number of which varies from 3 to 7, with a semicolon between adjacent words. The font of the English Abstract is Times New Roman, with the size of 12pt(small four).
% }

% \ekeywords{keyword 1, keyword 2, keyword 3, ……, keyword 7 (no punctuation at the end)}

% \makecover

% \clearpage


% !Mode:: "TeX:UTF-8"


\ctitle{槽道湍流带的稳定性和瞬态特性的数值模拟研究}  %封面用论文标题,自己可手动断行
\etitle{numerical study on stability and transient characteristics of  channel turbulent band}
\cfirstsubjecttitle{\textbf{一级学科}}
\cfirstsubject{\textbf{\underline{\makebox[14em][c]{数学}}}}   %专业
\csubjecttitle{\textbf{研究方向}}
\csubject{\textbf{\underline{\makebox[14em][c]{数学}}}}   %专业
\cauthortitle{\textbf{作者姓名}}     % 学位
\cauthor{\textbf{\underline{\makebox[14em][c]{伍昊洋}}}}   %学生姓名
\csupervisortitle{\textbf{指导教师}}
\csupervisor{\textbf{\underline{\makebox[14em][c]{宋保方}}}} %导师姓名

\teachertable{
\begin{table}[h]
\centering
\renewcommand{\arraystretch}{1}  % 临时定义一下行间距
\song\xiaosi{
\begin{tabularx}{\textwidth}{|*{4}{>{\centering\arraybackslash}X|}}
\hline
\textbf{答辩日期}                & \multicolumn{3}{c|}{20   年   月   日}       \\ \hline
\textbf{答辩委员会}               & \textbf{姓名} & \textbf{职称} & \textbf{工作单位} \\ \hline
\textbf{主席}                  &             &             &               \\ \hline
\multirow{2}{*}{\textbf{委员}} &             &             &               \\ \cline{2-4} 
                             &             &             &               \\ \hline
\end{tabularx}}
\end{table}}

\caffil{天津大学数学学院} %学院名称
\cdate{\CJKdigits{\the\year} 年\CJKnumber{\the\month} 月 \CJKnumber{\the\day} 日}
% 如需改成二〇一二年四月二十五日的格式,可以直接输入,即如下所示
\cdate{二〇二三年九月}
% \cdate{\the\year 年\the\month 月 \the\day 日} % 此日期显示格式为阿拉伯数字 如2012年4月25日


\declaretitle{独创性声明}
\declarecontent{
本人声明所呈交的学位论文是本人在导师指导下进行的研究工作和取得的研究成果,除了文中特别加以标注和致谢之处外,论文中不包含其他人已经发表或撰写过的研究成果,也不包含为获得 {\underline{\kaiGB{\sihao{\textbf{~~天津大学~~}}}}} 或其他教育机构的学位或证书而使用过的材料。与我一同工作的同志对本研究所做的任何贡献均已在论文中作了明确的说明并表示了谢意。
}
\authorizationtitle{学位论文版权使用授权书}
\authorizationcontent{
本学位论文作者完全了解{\underline{\kaiGB{\sihao{\textbf{~~天津大学~~}}}}}有关保留、使用学位论文的规定。特授权{\underline{\kaiGB{\sihao{\textbf{~~天津大学~~}}}}} 可以将学位论文的全部或部分内容编入有关数据库进行检索,并采用影印、缩印或扫描等复制手段保存、汇编以供查阅和借阅。同意学校向国家有关部门或机构送交论文的复印件和磁盘。
}
\authorizationadd{(保密的学位论文在解密后适用本授权说明)}
\authorsigncap{学位论文作者签名:}
\supervisorsigncap{导师签名:}
\signdatecap{签字日期:}

\cabstract{
湍流的转捩过程是研究湍流形成的一个重要课题。在转捩雷诺数下,槽道流动形成的湍流会以局部湍流带的形式存在。已有研究表明,湍流带有展向运动速度的下游端头对湍流带的形成和自维持起到关键的作用。由于湍流带存在一个展向运动速度,在物理实验研究中槽道侧壁的存在意味着湍流带必将与侧壁发生碰撞,而该问题虽有被提及但还没有被深入研究。特别地,关于充分发展的湍流带下游端头与槽道侧壁碰撞后,端头在边壁上是被吸收还是反射、湍流带是否能自持、是否存在湍流碰撞后衰减和存活的临界雷诺数等都是有待进一步深入研究的问题。

另一方面,现有的相关研究普遍认为湍流带在一定雷诺数以上可以自持。然而最近的研究结果揭示了湍流带的头部机制在维持长时间后会突然衰减,即湍流带有有限的寿命。在这一基础上需要对该问题进行进一步深入研究以验证相关理论的正确性。研究湍流带瞬态行为主要采用直接数值模拟方法(DNS),由于湍流带模拟需要极大的计算域以及湍流带极长的寿命,这需要消耗大量的时间与计算资源。本文尝试引入循环神经网络,希望能通过湍流带的历史数据预测湍流带的最终衰减,从而达到降低数值模拟的计算成本的目的。

对于以上情况,本文相关研究结果分为两部分:第一部分基于谱元法通过数值模拟实验探讨了在大宽高比带展向侧壁槽道中侧壁与湍流带的相互作用,发现了存在一临界值当雷诺数高于临界值时湍流带与侧壁碰撞后仍能存活,反之则会最终消散,该值所在的区间为$Re_{cr} \in [975,1000]$。通过数值实验得出进一步的推论:侧壁通过影响湍流带的头部速度型进而影响头部不稳定性最终导致头部在碰撞后消失。第二部分本文基于已有的应用神经网络预测流动瞬态性的研究,复现了该研究涉及的ESN神经网络并尝试将研究中提出的预测方法迁移到湍流带的寿命预测任务上。探讨了应用ESN、LSTM等神经网络在预测湍流带寿命上的潜力和局限性。目前ESN与LSTM网络均未能成功预测湍流带衰减过程,但LSTM网络比起ESN网络能更好地捕捉到湍流带的关键力学特征。
}

\ckeywords{湍流转捩,边壁效应,湍流带,寿命预测}

\eabstract{

The transition process of turbulence is an important topic in the study of turbulence formation. Under the transitional Reynolds number, the turbulent flow formed in the channel will exist in the form of local turbulent bands. Previous studies have shown that the downstream end with spanwise drift velocity of the turbulent band plays a crucial role in its self-sustaining and formation. Due to the presence of a spanwise drift velocity in the turbulent band, the existence of the channel sidewall in physical experiments implies that the turbulent band will inevitably collide with the sidewall. Although this problem has been mentioned, it has not been thoroughly studied. In particular, it is still necessary to further investigate issues such as whether the downstream end of the fully developed turbulent band is absorbed or reflected on the sidewall after collision, whether the turbulent band can be self-sustaining, whether there is turbulence attenuation and critical Reynolds number after collision, and survival, etc.

On the other hand, existing studies generally believe that the turbulent band can self-sustain above a certain Reynolds number. However, recent studies have revealed that the head mechanism of the turbulent band will suddenly decay after maintaining for a long time, indicating that the turbulent band has a finite lifetime. Based on this, it is necessary to further study this issue to verify the correctness of relevant theories. Direct numerical simulation (DNS) is mainly used to study the transient behavior of turbulent bands. Due to the need for a large computational domain and the extremely long lifetime of the turbulent band, this requires a lot of time and computational resources. This paper attempts to introduce recurrent neural networks to predict the final decay of the turbulent band based on its historical data, in order to reduce the computational cost of numerical simulations.

Based on the above situation, the research results of this paper are divided into two parts. The first part investigates the interaction between the side wall and the turbulent band in a high aspect ratio channel using spectral element method. It is found that there exists a critical value $Re_{cr} \in [975,1000]$ of Reynolds number, above which the turbulent band can survive after colliding with the side wall, while below which it will eventually dissipate. Numerical experiments further reveal that the side wall affects the velocity profile of the head of the turbulent band, leading to the instability of the head and eventual disappearance after colliding with the side wall. In the second part, based on the previous research achievements of applying neural networks to predict flow transients, the relevant neural network ESN is reproduced and the relevant predicting method is attempted to be applied to predict the lifetime of turbulent band. The potential and limitations of using neural network methods like ESN and LSTM to predict the lifetime of turbulent band are explored.
}

\ekeywords{transition, side wall, turbulent band, life prediction}

\makecover
\clearpage
